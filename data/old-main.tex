% Bastien Thouverez, Basile Fraboni
% Rendu 1 Analyse image
% Septembre 2016
% stylé
%%%%%%% %%% % % chère bîen

\section{Fonctionnalités du programme}

Sur la figure obtenue, via l'édition, il est possible de:
 \begin{document}
\item déplacer les points de contrôle sur la figure (ou à la main dans la fenêtre d'édition)
\item modifier/ajouter/supprimer un point de contrôle à la main (le degré s'adapte s'il n'y a plus assez de points)
\item changer le degré des morceaux de courbes (reste inchangé si 0 ou si supérieur au nombre de points de contrôle \item 1)
\item modifier le nombre de points affichés pour la courbe
\item de modifier les valeurs de la séquence nodale %% et là ça marche?
\item d'insérer un point dans la séquence nodale
\item d'afficher ou non les points de contrôle


\end{document}

