% Bastien Thouverez, Basile Fraboni
% Rendu 1 Analyse image
% Septembre 2016
% stylé
%%%%%%% %%% % % chère bîen

\section{Fonctionnalit\'es du programme}

Sur la figure obtenue, via l'\'edition, il est possible de:
 \begin{document}
\item d\'eplacer les points de contr\^ole sur la figure (ou \`a la main dans la fen\^etre d'\'edition)
\item modifier/ajouter/supprimer un point de contr\^ole \`a la main (le degr\'e s'adapte s'il n'y a plus assez de points)
\item changer le degr\'e des morceaux de courbes (reste inchang\'e si 0 ou si sup\'erieur au nombre de points de contr\^ole \item 1)
\item modifier le nombre de points affich\'es pour la courbe
\item de modifier les valeurs de la s\'equence nodale %% et là ça marche?
\item d'ins\'erer un point dans la s\'equence nodale
\item d'afficher ou non les points de contr\^ole


\end{document}

