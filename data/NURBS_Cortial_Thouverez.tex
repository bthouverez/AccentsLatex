% Bastien Thouverez, Yann Cortial
% Rendu 2 Modélisation géométrique
% Octobre 2016


\documentclass[10pt]{article}
\usepackage[french]{babel}
\usepackage[T1]{fontenc}
\usepackage[utf8]{inputenc}
\usepackage{a4wide}
\usepackage{fancyhdr}
\usepackage{algorithmic}
\usepackage{algorithm}
\pagestyle{fancyplain}
\usepackage{graphicx}
\usepackage{wrapfig}
\usepackage{caption}
\usepackage{pgf,tikz}
\usepackage{mathrsfs}
\usepackage{hyperref}
\usepackage{todonotes}
\usepackage{movie15}
\usetikzlibrary{arrows}


\setcounter{tocdepth}{2}
\addtolength{\headheight}{1.5pt}
\fancyfoot[LO]{Universit\'e Lyon1} 



\begin{document}
\begin{titlepage}
	\centering
	{\includegraphics[width=150px]{lyon1.png}\par}
	\vspace{.5cm}
	{\scshape\large Universit\'e Claude Bernard, Lyon1 \par}
	\vspace{2cm}
	{\scshape\LARGE Mod\'elisation g\'eom\'etrique \par}
	\vspace{1cm}
	{\huge\bfseries \Huge{Travaux Pratiques 2\\ NURBS via NanoEdit\par}}
	\vfill
	{\Large\itshape Yann Cortial, Bastien Thouverez\par}

\end{titlepage}



\section{Lancement du programme}
Notre programme NURBS peut \^etre lanc\'e sur NanoEdit via le raccourci 'YB'

ou via le menu Cr\'eation/BabsYann/NURBS\_insertion


\section{Fonctionnalit\'es du programme}

Sur la figure obtenue, via l'\'edition, il est possible de:
 \begin{itemize}
\item d\'eplacer les points de contr\^ole sur la figure (ou \`a la main dans la fen\^etre d'\'edition)
\item modifier/ajouter/supprimer un point de contr\^ole \`a la main (le degr\'e s'adapte s'il n'y a plus assez de points)
\item changer le degr\'e des morceaux de courbes (reste inchang\'e si 0 ou si sup\'erieur au nombre de points de contr\^ole \item 1)
\item modifier le nombre de points affich\'es pour la courbe
\item de modifier les valeurs de la s\'equence nodale
\item d'ins\'erer un point dans la s\'equence nodale
\item d'afficher ou non les points de contr\^ole
 
 \end{itemize}

\begin{figure}[!h]
\centering
\includegraphics[width=200px]{edition.png}
\caption{Panneau d'\'edition}
\end{figure}


\section{R\'esultat en images}

Un gif est pr\'esent dans l'archive. Il montre l'insertion de n\oe ud dans la s\'equence nodale.

A partir de la courbe de d\'epart, 13 points sont ajout\'es un \`a un dans la s\'equence nodale.

La courbe reste bien inchang\'ee.\\

La s\'equence nodale est pass\'ee de\\

(0 0 0 0.2 0.4 0.6 0.8 1 1 1)\\

\`a \\

(0 0 0 0.1 0.2 0.3 0.4 0.5 0.6 0.7 0.72 0.73 0.74 0.75 0.77 0.78 0.79 0.8 0.9 0.95 0.98 1 1 1)

\begin{figure}[!h]
\centering
\includegraphics[width=400px]{insert1.png}\vspace{.4cm}
\includegraphics[width=400px]{insert2.png}
\caption{Courbe avant et apr\`es insertions de points dans la s\'equence nodale}
\end{figure}


\end{document}

