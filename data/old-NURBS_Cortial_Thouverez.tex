% Bastien Thouverez, Yann Cortial
% Rendu 2 Modélisation géométrique
% Octobre 2016


\documentclass[10pt]{article}
\usepackage[french]{babel}
\usepackage[T1]{fontenc}
\usepackage[utf8]{inputenc}
\usepackage{a4wide}
\usepackage{fancyhdr}
\usepackage{algorithmic}
\usepackage{algorithm}
\pagestyle{fancyplain}
\usepackage{graphicx}
\usepackage{wrapfig}
\usepackage{caption}
\usepackage{pgf,tikz}
\usepackage{mathrsfs}
\usepackage{hyperref}
\usepackage{todonotes}
\usepackage{movie15}
\usetikzlibrary{arrows}


\setcounter{tocdepth}{2}
\addtolength{\headheight}{1.5pt}
\fancyfoot[LO]{Universit\'e Lyon1} 



\begin{document}
\begin{titlepage}
	\centering
	{\includegraphics[width=150px]{lyon1.png}\par}
	\vspace{.5cm}
	{\scshape\large Universit\'e Claude Bernard, Lyon1 \par}
	\vspace{2cm}
	{\scshape\LARGE Modélisation géométrique \par}
	\vspace{1cm}
	{\huge\bfseries \Huge{Travaux Pratiques 2\\ NURBS via NanoEdit\par}}
	\vfill
	{\Large\itshape Yann Cortial, Bastien Thouverez\par}

\end{titlepage}



\section{Lancement du programme}
Notre programme NURBS peut être lancé sur NanoEdit via le raccourci 'YB'

ou via le menu Création/BabsYann/NURBS\_insertion


\section{Fonctionnalités du programme}

Sur la figure obtenue, via l'édition, il est possible de:
 \begin{itemize}
\item déplacer les points de contrôle sur la figure (ou à la main dans la fenêtre d'édition)
\item modifier/ajouter/supprimer un point de contrôle à la main (le degré s'adapte s'il n'y a plus assez de points)
\item changer le degré des morceaux de courbes (reste inchangé si 0 ou si supérieur au nombre de points de contrôle \item 1)
\item modifier le nombre de points affichés pour la courbe
\item de modifier les valeurs de la séquence nodale
\item d'insérer un point dans la séquence nodale
\item d'afficher ou non les points de contrôle
 
 \end{itemize}

\begin{figure}[!h]
\centering
\includegraphics[width=200px]{edition.png}
\caption{Panneau d'édition}
\end{figure}


\section{Résultat en images}

Un gif est présent dans l'archive. Il montre l'insertion de n\oe ud dans la séquence nodale.

A partir de la courbe de départ, 13 points sont ajoutés un à un dans la séquence nodale.

La courbe reste bien inchangée.\\

La séquence nodale est passée de\\

(0 0 0 0.2 0.4 0.6 0.8 1 1 1)\\

à \\

(0 0 0 0.1 0.2 0.3 0.4 0.5 0.6 0.7 0.72 0.73 0.74 0.75 0.77 0.78 0.79 0.8 0.9 0.95 0.98 1 1 1)

\begin{figure}[!h]
\centering
\includegraphics[width=400px]{insert1.png}\vspace{.4cm}
\includegraphics[width=400px]{insert2.png}
\caption{Courbe avant et après insertions de points dans la séquence nodale}
\end{figure}


\end{document}

